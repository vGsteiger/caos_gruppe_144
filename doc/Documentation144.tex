\documentclass[12pt,a4paper]{article}
\usepackage{graphicx}
\usepackage{xcolor}
\usepackage[margin=0.8in]{geometry}
\usepackage{titling}
\usepackage{indentfirst}
\usepackage{wrapfig}
\usepackage{amsfonts}
\usepackage{tocloft}
\usepackage{tabularx}
\renewcommand{\cftsecleader}{\cftdotfill{\cftdotsep}}
\setlength{\parindent}{0em}
\title{\textbf{CAOS Projekt : 2 x 6 x 12 LED RGB Quader}}
\author{Joey Zgraggen, Moritz Würth, Viktor Gsteiger}

\begin{document}
\renewcommand\contentsname{Inhaltsverzeichnis}
\begin{titlepage}
\maketitle
TODO: BILD VOM PROJEKTERZEUGNIS
\end{titlepage}
\tableofcontents
\newpage

\section{Ziel des Projektes}

Die Grundidee von unserem Projekt war es zuerst einen 5 x 5 x 5 LED RGB Würfel zu bauen, jedoch haben wir uns dann im Prozess der Entscheidungsfindung
dazu entschieden, etwas anderes zu bauen was sich vom "typischen" LED RGB Würfel unterscheidet.
Deshalb haben wir die Dimensionen ein wenig angepasst, sodass wir am Ende ein Quader haben werden. Damit wird es uns immer noch möglich sein gute 3D-Effekte
anzeigen zu können. Die Anordnung der LED's kann als Matrix verstanden werden mit entsprechenden boolean Variablen, welche beim entsprechenden LED die erwünschte Farbe
anzeigen. 
Das Projekt lässt einen grossen Spielraum bezüglich Ideen zu. Sollte man also selbst Ideen haben, welche sich gut zu unserem Projekt ergänzen würden, kann man diese ohne weitere
Probleme implementieren.


\section{Verwendete Materialien}

Im Folgenden ist eine Auflistung der verschiedenen Materialien, welche für das Projekt verwendet wurden.

\subsection{RGB LED's}

Um den LED-Quader ein wenig bunter gestalten zu können wurden RGB's verwendet.:

\begin{itemize}
    \item LED RGB Common Cathode 4-Pin F5 5MM Diode
\end{itemize}

\subsection{Schieberegister}

Schieberegister erweisen sich für ein solches Projekt als äusserst nützlich, um die vielen Pins der einzelnen RGB LED's ansprechen zu können. \\

Wir haben uns dabei an schon bereits existierende Projekte orientiert und uns für die 74HC595 Schieberegister entschieden.

\subsection{Transistoren}

TODO: npn PN2222 Transistoren

\subsection{Sensoren}

Für das Projekt wurden zusätzlich Sensoren verwendet, um weitere Features zu gewährleisten.

Folgende Sensoren stehen zur Auswahl:
\begin{itemize}
    \item Temperatursensoren:
    \item Luftfeuchtigkeitssensoren:
\end{itemize}

\section{Materialkosten}

\begin{tabularx}{\textwidth}{p{0.25\textwidth} | l | l | l | r |}
    \textbf{Produkt} & \textbf{Menge} & \textbf{Preis} & \textbf{Total} \\
    \cline{1-4}
    RGB LED's & 144x & .- & .- \\
    \cline{1-4}
    RGB LED's & 144x & .- & .- \\
    \cline{1-4}
    RGB LED's & 144x & .- & .- \\
    \cline{1-4}
    RGB LED's & 144x & .- & .- \\
    \cline{1-4}
    RGB LED's & 144x & .- & .- \\
    
\end{tabularx}

\section{Design}

\subsection{}


\section{Features}

\subsection{Equalizer}

\subsection{\textit{Eventuell weitere Features (kommt noch)}}

\subsection{Sensoren}

\subsubsection{Wärmesensoren}

\subsubsection{Luftfeuchtigkeitssensoren}

\section{Aufbau des Quaders}

\subsection{Quader}

\subsection{Transistoren}

\subsection{Schieberegister}

\subsection{Probleme beim Aufbau}

\section{Zusammenfassung}

\newpage

\textbf{Quellen}:

\begin{itemize}
    \item 
\end{itemize}










\end{document}